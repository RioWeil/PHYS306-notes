\documentclass{article}
\usepackage[margin=3pt,landscape]{geometry}
\usepackage[utf8]{inputenc}
\usepackage[T1]{fontenc}
\usepackage{multicol}
\usepackage{amsmath} % AMS Math Package
\usepackage{amssymb}  % Math symbols such as \mathbb
\usepackage{commath}
\usepackage{tensor}
\usepackage{mathtools}
\usepackage{physics}
\usepackage{bm}
\usepackage{graphicx}
\usepackage{siunitx}
\usepackage[nopar]{lipsum}%just to generate text for the example

\let\vaccent=\v % rename builtin command \v{} to \vaccent{}
\renewcommand{\v}[1]{\ensuremath{\mathbf{#1}}} % for vectors
\newcommand{\gv}[1]{\ensuremath{\mbox{\boldmath$ #1 $}}} 
% for vectors of Greek letters
\newcommand{\uv}[1]{\ensuremath{\mathbf{\hat{#1}}}} % for unit vector
\newcommand{\avg}[1]{\left< #1 \right>} % for average
\newcommand{\pdc}[3]{\left( \frac{\partial #1}{\partial #2}
	\right)_{#3}} % for thermodynamic partial derivatives
\let\baraccent=\= % rename builtin command \= to \baraccent
\newcommand{\divides}{\phantom{i} \arrowvert \phantom{i}}
\newcommand{\ndiv}{\nmid}
\newcommand{\bicon}{\Leftrightarrow}
\newcommand{\imply}{\Rightarrow}
\newcommand{\sol}{\\ \\ \emph{Solution.}}
\newcommand{\soln}{\emph{Solution.}}
\newcommand{\pset}[1]{\mathcal{P}({#1})}
\newcommand{\rel}{\mathrel{R}}
\newcommand{\nrel}{\not \mathrel{R}}
\newcommand{\Arg}[1]{\mathrm{Arg}\left({#1}\right)}
\newcommand{\limi}[1]{\lim_{#1 \rightarrow \infty}}
\newcommand{\limni}[1]{\lim_{#1 \rightarrow -\infty}}
\newcommand{\rie}{\mathcal{R}}



\renewcommand{\=}[1]{\stackrel{#1}{=}} % for putting numbers above 
\renewcommand{\emptyset}{\varnothing}

%\dif is differential symbol
%\od[n-th deriv]{f}{x}, n-th deriv can be left out for 1
%\pd does same thing as above, for partial derivatives
%adding t (ex. \tpd) will make it text style, d (ex. \dpd) will make it display style
%\md will do mixed derivatives. ex. \md{f}{5}{x}{2}{y}{3}
%\eval[optional argument]{first argument} is a command for the notation of an expression denoted by the first argument evaluated at a particular condi- tion. The value for the optional argument ranges from 0 to 4 with higher values resulting in larger delimiters. The default value for the optional argument is -1 which results in automatic sizing for the delimiters. For example, \eval{f(\epsilon)}_{\epsilon=0}
%\enVert}[optional argument]{first argument}is a command which enclose the argument in double vert-bar delimiters.
%\fullfunction{first argument}....{fifth argument}isacommandwhich nicely formats a function. The first argument denotes the function name, the second the domain and the third the image of the function. The forth is the parameter which is mapped to the expression denoted by argument five. E.g.:\fullfunction{f}{\mathbb R}{\mathbb R}{x}{\sqrt{x}}
\newcommand{\xhat}{\hat{\v{x}}}
\newcommand{\yhat}{\hat{\v{y}}}
\newcommand{\zhat}{\hat{\v{z}}}
\newcommand{\rhat}{\hat{\v{r}}}
\newcommand{\shat}{\hat{\v{s}}}
\newcommand{\phihat}{\hat{\bm{\phi}}}
\newcommand{\thetahat}{\hat{\bm{\theta}}}
\newcommand{\ehat}{\hat{\v{e}}}

\newcommand{\LL}{\mathcal{L}}
\newcommand{\HH}{\mathcal{H}}

\providecommand{\wave}[1]{\v{\tilde{#1}}}
\providecommand{\fr}{\frac}
\providecommand{\RR}{\mathbb{R}}
\providecommand{\NN}{\mathbb{N}}
\providecommand{\ZZ}{\mathbb{Z}}
\providecommand{\QQ}{\mathbb{Q}}
\providecommand{\CC}{\mathbb{C}}
\providecommand{\MM}{\mathbb{M}}
\providecommand{\KK}{\mathbb{K}}
\providecommand{\II}{\mathbb{I}}
\providecommand{\DD}{\mathbb{D}}
\providecommand{\seq}{\subseteq}
\providecommand{\e}{\epsilon}
\newcommand\m[1]{\begin{bmatrix}#1\end{bmatrix}}

% redefinition of \normalsize to use \footnotesize and decrease the values for
% \abovedisplyskip, \belowdisplayskip, and the "short" variants
\makeatletter
\renewcommand\normalsize{%
	\@setfontsize\footnotesize\@viiipt{8}%
	\abovedisplayskip 1\p@ \@plus\p@ \@minus\p@
	\belowdisplayskip 1\p@ \@plus\p@ \@minus\p@
	\abovedisplayshortskip \z@ \@plus\p@
	\belowdisplayshortskip \z@ \@plus\p@ \@minus\p@
	\let\@listi\@listI}
\makeatother

\newcommand{\bb}[1]{\mathbb{#1}} %Bold face math letters
\newcommand{\half}[0]{\frac{1}{2}} %for 1/2
\newcommand{\ra}[0]{\rightarrow} %for ->
\newcommand{\ora}[1]{\overrightarrow{#1}} %for vectors with arrows

\pagestyle{empty}

% multicol parameters
\setlength\premulticols{1pt}
\setlength\postmulticols{1pt}
\setlength\multicolsep{1pt}
\setlength\columnsep{2pt}
\raggedright

\begin{document}
	\begin{multicols}{4}
		Kinematics in Polar Coordinates:
		\begin{align}
			\begin{split}
				\v{v}=\dot{r}\hat{r}+r\dot{\phi}\hat{\phi} \\
				\v{a}=(\ddot{r}-r\dot{\phi}^2)\hat{r}+(r\ddot{\phi}+2\dot{r}\dot{\phi})\hat{\phi}
			\end{split}
		\end{align}
		Lagrange Multipliers (for a system with $m$ holonomic constraints):
		\begin{equation}
			\dpd{\LL}{q_i} + \sum_{k=1}^m\lambda_k \dpd{f_k}{q_i} = \dod{}{t}\dpd{\LL}{\dot{q}_i}
		\end{equation}
		General solution to simple harmonic oscillator:
		\begin{equation}
			x(t) = A\cos(\omega_0 t - \delta), \quad \omega_0 = \sqrt{\frac{k}{m}}
		\end{equation}
		Damped Oscillator ODE:
		\begin{equation}
			\ddot{x} + 2\beta\dot{x} + \omega_0^2x = 0, \quad \beta = \frac{b}{2m}
		\end{equation}
		Overdamped Solution:
		\begin{equation}
			\begin{split}
				x(t) &= C_1\exp(r_1t) + C_2\exp(r_2t) \\
				&= C_1\exp(\left(-\beta + \sqrt{\beta^2 - \omega_0^2}\right)t) + \\
				&\quad C_2\exp(\left(-\beta - \sqrt{\beta^2 - \omega_0^2}\right)t)
			\end{split}
		\end{equation}
		Critically Damped solution:
		\begin{equation}
			x(t) = C_1\exp(-\beta t) + C_2t\exp(-\beta t)
		\end{equation}
		Underdamped Solution:
		\begin{equation}
			x(t) = A\exp(-\beta t)\cos(\left(\sqrt{\omega_0^2 - \beta^2}\right)t - \delta)
		\end{equation}
		Damped Driven Oscillator ODE:
		\begin{equation}
			\ddot{z} + 2\beta\dot{z} + \omega_0^2z = f_0\exp(i\omega t)
		\end{equation}
		Amplitude of solution $C\exp(i\omega t)$:
		\begin{equation}
			C = \frac{f_0}{-\omega^2 + 2i\beta \omega + \omega_0^2} = A\exp(i\delta)
		\end{equation}
		Amplitude squared and phase:
		\begin{equation}
			\begin{split}
				A^2 &= CC^* = \frac{f_0^2}{(\omega_0^2 - \omega^2)^2 + 4\beta^2\omega^2} \\
				&\quad \delta = \arctan(\frac{2\beta \omega}{\omega_0^2 - \omega^2})
			\end{split}
		\end{equation}
		General solution to $\ddot{x} + 2\beta\dot{x} + \omega_0^2x + f_0\cos(\omega t)$:
		\begin{equation}
			\begin{split}
				x(t) &= A\cos(\omega t - \delta) + C_1\exp(r_1 t) + C_2\exp(r_2t) \\
				&= x_{periodic}(t) + x_{trans}(t)
			\end{split}
		\end{equation}
		Q factor:
		\begin{equation}
			Q = \frac{\omega_0}{2\beta}
		\end{equation}
		General Coupled Oscillator Matrix equation:
		\begin{equation}
			\MM\ddot{\v{x}} = -\KK\v{x}
		\end{equation}
		Trial Solution:
		\begin{equation}
			\v{z}=\m{a_1 \\a_2}=\v{a}\exp(i\omega t)
		\end{equation}
		Eigenfrequency characteristic equation:
		\begin{equation}
			\det(\KK - \omega^2\MM) = 0
		\end{equation}
		Eigenmodes:
		\begin{equation}
			(\KK - \omega\MM)\m{A_1 \\A_2} = \v{0}
		\end{equation}
		Normal coordinates:
		\begin{equation}
			\v{q}(t) = \sum_{i=1}^n\v{a}_i\xi_i(t), \quad \text{such that } \ddot{\xi}_i + \omega_i^2\xi_i = 0
		\end{equation}
		Newton's law for linearly accelerating frame:
		\begin{equation}
			m\ddot{\v{r}} = \v{F} - m\v{A}
		\end{equation}
		Velocity in rotating frames
		\begin{equation}
			\v{v} = \dot{\v{r}} = \bm{\omega} \times \v{r}
		\end{equation}
		Addition of angular velocities:
		\begin{equation}
			\bm{\omega}_{31} = \bm{\omega}_{32} + \bm{\omega}_{21}
		\end{equation}
		Time derivatives in non-inertial frames:
		\begin{equation}
			\left(\dod{\v{Q}}{t}\right)_{S_0} = \left(\dod{\v{Q}}{t}\right)_{S} + \bm{\Omega} \times \v{Q}
		\end{equation}
		Newton's Law in Rotating Frame:
		\begin{equation}
			\begin{split}
				m\ddot{\v{r}} &= \v{F} + \v{F}_{cor} + \v{F}_{cent} + \v{F}_{euler} \\
				&= \v{F} + 2m\dot{\v{r}}\times \bm{\Omega} + m(\bm{\Omega} \times \v{r})\times \bm{\Omega} + m\v{r} \times \dot{\bm{\Omega}}
			\end{split}
		\end{equation}
		Position of COM:
		\begin{equation}
			\v{R} = \frac{1}{M}\sum_\alpha m_\alpha\v{r}_\alpha
		\end{equation}
		Momentum of COM:
		\begin{equation}
			\v{P} = M\dot{\v{R}} = \sum_\alpha m_\alpha \dot{\v{r}}_\alpha
		\end{equation}
		External force:
		\begin{equation}
			\v{F}_{ext} = M\ddot{\v{R}}
		\end{equation}
		Angular momentum of spinning rigid bodies:
		\begin{equation}
			\begin{split}
				\v{L} &= \sum_\alpha (\v{R} \times m_\alpha\dot{\v{R}}) + \sum_\alpha (\v{r}'_\alpha + m_\alpha\dot{\v{r}}') \\
				&= \v{L}_{orbital} + \v{L}_{spin}
			\end{split}
		\end{equation}
		Potential energy:
		\begin{equation}
			U = U_{ext} + U_{int} = U_{ext} + \sum_{i<j}U_{ij}(r_{ij})
		\end{equation}
		Kinetic energy:
		\begin{equation}
			T = \frac{1}{2}\sum_\alpha m_\alpha \dot{\v{r}}^{'\alpha}_2
		\end{equation}
		Angular momentum for rotation about z axis:
		\begin{equation}
			\begin{split}
				\v{L} &= \m{L_x \\ L_y \\ L_z} \\
				&= \sum_\alpha (\v{r}_{\alpha} \times m_\alpha\v{v}_\alpha) \\
				&= \sum_\alpha (\v{r}_{\alpha} \times m_\alpha(\omega\zhat \times \v{r}_\alpha)) \\
				&= \sum_\alpha m_\alpha\omega\m{-z_\alpha x_\alpha \\ -z_\alpha y_\alpha \\ x_\alpha^2 + y_\alpha^2} \\
				&= \m{I_x \omega \\ I_{y}\omega \\ I_z \omega}
			\end{split}
		\end{equation}
		Inertia Tensor:
		\begin{equation}
			\v{L} = \II \bm{\omega} = \m{I_{xx} & I_{xy} & I_{xz} \\ I_{yx} & I_{yy} & I_{yz} \\ I_{zx} & I_{zy} & I_{zz}}\bm{\omega}
		\end{equation}
		Inertia Tensor entries (discrete):
		\[\II = \sum_\alpha m_\alpha \m{(y_\alpha^2 + z_\alpha^2) & - x_\alpha y_\alpha & -x_\alpha z_\alpha
			\\ -y_\alpha x_\alpha & (z_\alpha^2 + x_\alpha^2) & -y_\alpha z_\alpha
			\\ -z_\alpha x_\alpha & -z_\alpha y_\alpha & (x_\alpha^2 + y_\alpha^2)}\]
		Inertia Tensor entries (continuous)
		\[\II = \int dV\rho(x, y, z)\m{y^2 + z^2 & -xy & -xz \\ -xy & x^2 + z^2 & -yz \\ -xz & -yz & x^2 + y^2
		}\]
		Inertia Tensor entries (index notation):
		\begin{equation}
			I_{ij} = \int dV\rho(x, y, z)\left(\v{r}^2\delta_{ij} - r_ir_j\right)
		\end{equation}
		Parallel Axis Theorem ($\v{a}$ is vector from center):
		\begin{equation}
			J_{ij} = I_{ij} + M(a^2\delta_{ij} - a_ia_j)
		\end{equation}
		Principle Axis:
		\begin{equation}
			\v{L} = \m{\lambda_1 & 0 & 0 \\ 0 & \lambda_2 & 0 \\ 0 & 0 & \lambda_3}\m{\omega_1 \\ \omega_2 \\ \omega_3} = \m{\lambda_1\omega_1 \\ \lambda_2\omega_2 \\ \lambda_3\omega_3}
		\end{equation}
		Torque:
		\begin{equation}
			\bm{\Gamma} = \dot{\v{L}}
		\end{equation}
		Euler's Equations:
		\begin{equation} \label{eq1}
			\begin{split}
				\Gamma_1 &= \lambda_1\dot{\omega}_1 - (\lambda_2 - \lambda_3)\omega_2\omega_3 \\
				\Gamma_2 &= \lambda_2\dot{\omega}_2 - (\lambda_3 - \lambda_1)\omega_1\omega_3\\
				\Gamma_3 &= \lambda_3\dot{\omega}_3 - (\lambda_1 - \lambda_2)\omega_1\omega_2
			\end{split}
		\end{equation}
		Taylor Expansion of $f(x)$ About $a$:
		\begin{equation}
			f(x)=f(a)+\frac {f'(a)}{1!} (x-a)+ \frac{f''(a)}{2!} (x-a)^2+\mathcal{O}(x^3)
		\end{equation}
		Small-Angle Approximation About Equilibrium Point:
		\begin{equation} \label{eq1}
			\begin{split}
				\cos(\theta_0+\epsilon)\approx\cos\theta_0-\epsilon\sin\theta_0 \\
				\sin(\theta_0+\epsilon)\approx\sin\theta_0+\epsilon\cos\theta_0
			\end{split}
		\end{equation}
		Miscellaneous Trig. Identities:
		\begin{align}
			\sin(a+b)=\sin(a)\cos(b)+\cos(a)\sin(b) \\
			\cos(a+b)=\cos(a)\cos(b)-\sin(a)\sin(b) \\
			\sin(a-b)=\sin(a)\cos(b)-\cos(a)\sin(b) \\
			\cos(a-b)=\cos(a)\cos(b)+\sin(a)\sin(b)
		\end{align}
	\end{multicols}
\end{document}
