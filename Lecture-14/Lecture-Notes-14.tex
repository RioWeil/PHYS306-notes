\section{Lecture 14}
\subsection{Lecture Notes - Review of Inertial Frames}
\subsubsection{Definition \& Newton's Law in an accelerating Frame}
By definition, an inertial frame $S_0$ is a frame with no acceleration or rotation. We now consider a constantly accelerating frame $S$ with $\v{A} = \dot{\v{v}}$. Newton's law holds in the inertial frame, so:
\[m\ddot{\v{r}}_0 = \v{F}\]
The velocity $\dot{\v{r}}_0$ (in the non-inertial frame) and $\dot{\v{r}}$ (in the inertial frame) are related by:
\[\dot{\v{r}}_0 = \dot{\v{r}} + \v{v}\]
Taking the time derivative:
\[\ddot{\v{r}} = \ddot{\v{r}}_0 - \v{A}\]
So applying Newton's law:
\[m\ddot{\v{r}} = \v{F} - m\v{A} = \v{F} + \v{F}_{inertial}\]
We get an extra term accounting for the fictitous force.