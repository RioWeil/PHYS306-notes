\documentclass[../PHYS306Notes.tex]{subfiles}

\begin{document}
\section{Lecture 2}
\subsection{Lecture Notes - Lagrangian Mechanics Part 2}
\subsubsection{Review of Variational Principle}
Variational principle. We want to minimize a functional (that takes in a function as an argument), by finding stationary points. This is reminiscent of the process of finding extremum in elementary/single variable calculus, where if we move slightly away from the point, we don't change the function very much. Now, we generalize this notion to a function space, where if we change the function slightly, we do not change the functional very much.

\noindent We want to derive that the shortest path between two points is a straight line (no objections general relativists!). We consider the variation $y(x) + \alpha\eta(x)$ around the true path $y(x)$. Look at worksheet one for the solutions/walkthrough of this.
\end{document}