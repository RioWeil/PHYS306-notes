\section{Lecture 6}
\subsection{Lecture Notes - Charged Particles in EM Fields}
\subsubsection{Newton's Law in the Presence of EM Fields}
Newton says:
\[m\ddot{\v{r}} = q(\v{E} + \dot{\v{r}}\times \v{B})\] for an electric field $\v{E}$ and magnetic field $\v{B}$. This is just the sum of the Coloumb force and Lorentz force.

\subsubsection{Scalar and Vector Potentials}
We will now apply the ideas of vector potential (for the magnetic field) from PHYS 301 to this situation. Recall that we defined the vector potential $\v{A}$ as \[\v{B} = \curl{\v{A}}\].
We also recall Faraday's Law, which states (combined with the definition of the potential above) that:
\[\curl{\v{E}} = -\dpd{\v{B}}{t} = -\dpd{\curl{\v{A}}}{t}\]
Interchanging the order of taking the curl and taking the time derivative, we see that:
\[\curl{\left(\v{E} + \dpd{\v{A}}{t}\right)} = \v{0}\]
So there exists a scalar potential $V$ such that:
\[-\grad{V} = \v{E} + \dpd{\v{A}}{t}\]
Which comes from the fact that a curl of a gradient is zero. We can rewrite this to say that:
\[\v{E} = -\grad{V} - \dpd{\v{A}}{t}\]
Now, we construct $\LL$ such that we get the Lorentz Force. Generalize $U = qV$ to get:
\[U' = qV - q\dot{\v{r}}\cdot\v{A}\]
And we use $\LL' = T - U'$ as our Lagrangian.

\subsubsection{Conservation of Angular Momentum in Spherically Symmetric Lagrangian}
This central force Lagrangian $\LL=\frac12 \mu |\dot{\v{r}}|^2-U(r)$ is spherically symmetric (ie. doesn't change when the coordinate system rotates). This implies energy is not always conserved but angular momentum is conserved. To see why, we will study rotational symmetry around z-axis.
When a vector in 3D space $\m{x \\ y \\ z}$ is rotated about z by $\theta$, the new vector is
\[\m{x' \\ y' \\ z'}=\m{\cos\theta & -\sin\theta & 0 \\ \sin\theta & \cos\theta & 0 \\ 0 & 0 & 1}\m{x \\ y \\ z}\]
Consider infinitesimal rotation by $\delta\theta$ and using small angle approximations $\cos\theta\approx1$ and $\sin\theta\approx\theta$,
\[\m{x' \\ y' \\ z'}=\m{1 & -\delta\theta & 0 \\ \delta\theta & 1 & 0 \\ 0 & 0 & 1}\m{x \\ y \\ z}\]
The Lagrangian becomes
\[\LL'(x',y',z',\dot{x'},\dot{y'},\dot{z'})=\LL(x-\delta\theta y,y+\delta\theta x,z,\dot{x}-\delta\theta \dot{y},\dot{y}+\delta\theta \dot{x},\dot{z})\]
Applying the first order multivariable Taylor expansion about $(x,y,z,\dot{x},\dot{y},\dot{z})$ gives
\[\LL'(x',y',z',\dot{x'},\dot{y'},\dot{z'})=\LL(x,y,z,\dot{x},\dot{y},\dot{z})+\dpd{\LL}{x'}(x'-x)+\dpd{\LL}{y'}(y'-y)+\dpd{\LL}{z'}(z'-z)+\dpd{\LL}{\dot{x'}}(\dot{x'}-\dot{x})+\dpd{\LL}{\dot{y'}}(\dot{y'}-\dot{y})+\dpd{\LL}{\dot{z'}}(\dot{z'}-\dot{z})\]
To calculate a x-term, we know that $x'=x-\delta\theta y$. This gives $x'-x=-\delta\theta y$ and $\dpd{\LL}{x'}=\dpd{\LL}{x}\dpd{x}{x'}=\dpd{\LL}{x}$. The other terms can be calculated the similar way. All of these gives
\[\LL'(x',y',z',\dot{x'},\dot{y'},\dot{z'})=\LL(x,y,z,\dot{x},\dot{y},\dot{z})-\delta\theta \mathbin{\textcolor{blue}{y\dpd{\LL}{x}}}+\delta\theta \mathbin{\textcolor{red}{x\dpd{\LL}{y}}}-\delta\theta \mathbin{\textcolor{blue}{\dot{y}\dpd{\LL}{\dot{x}}}}+\delta\theta\mathbin{\textcolor{red}{\dot{x}\dpd{\LL}{\dot{y}}}}\]
\[\LL'(x',y',z',\dot{x'},\dot{y'},\dot{z'})=\LL(x,y,z,\dot{x},\dot{y},\dot{z})+\delta\theta \left[ \left( \mathbin{\textcolor{red}{x\dpd{\LL}{y}}} - \mathbin{\textcolor{blue}{y\dpd{\LL}{x}}} \right)+\left( \mathbin{\textcolor{red}{\dot{x}\dpd{\LL}{\dot{y}}}} - \mathbin{\textcolor{blue}{\dot{y}\dpd{\LL}{\dot{x}}}} \right) \right]\]
Because of rotational symmetry, Lagrangian should not change at all (ie. $\LL'-\LL=0$). And this means
\[\left( \mathbin{\textcolor{red}{x\dpd{\LL}{y}}} - \mathbin{\textcolor{blue}{y\dpd{\LL}{x}}} \right)+\left( \mathbin{\textcolor{red}{\dot{x}\dpd{\LL}{\dot{y}}}} - \mathbin{\textcolor{blue}{\dot{y}\dpd{\LL}{\dot{x}}}} \right)=0\]
Substituting EL equations $\dpd{\LL}{x}=\dod{}{t}\left(\dpd{\LL}{\dot{x}}\right)$ and $\dpd{\LL}{y}=\dod{}{t}\left(\dpd{\LL}{\dot{y}}\right)$,
\[ \mathbin{\textcolor{red}{x\dod{}{t}\left(\dpd{\LL}{\dot{y}}\right)}} + \mathbin{\textcolor{red}{\dod{x}{t}\dpd{\LL}{\dot{y}}}} - \mathbin{\textcolor{blue}{y\dod{}{t}\left(\dpd{\LL}{\dot{x}}\right)}} - \mathbin{\textcolor{blue}{\dod{y}{t}\dpd{\LL}{\dot{x}}}}=0\]
\[ \dod{}{t}\left( \mathbin{\textcolor{red}{x\dpd{\LL}{\dot{y}}}} - \mathbin{\textcolor{blue}{y\dpd{\LL}{\dot{x}}}} \right)=0\]
Denote $\dpd{\LL}{\dot{y}}=p_y$ and $\dpd{\LL}{\dot{x}}=p_x$
\[ \dod{}{t}\left( \mathbin{\textcolor{red}{xp_y}} - \mathbin{\textcolor{blue}{yp_x}} \right)=0\]
Notice that $xp_y-yp_x$ is z-component of angular momentum $\v{L}$. Therefore
\[ \dod{L_z}{t}=0\]
So angular momentum is conserved under rotational symmetry. We will discuss more about this next class.