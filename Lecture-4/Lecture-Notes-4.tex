\section{Lecture 4}
\subsection{Lecture Notes - Lagrangian Mechanics Part 4}
\subsubsection{Proof of Lagrange Equations for Holonomic Systems}
We consider a holonomic system with $f(\v{r}_i, t) = 0$ constraints. Consider a true path $\v{r}^*(t)$, and a general path with variation of $\v{r}(t) = \v{r}^*(t) + \alpha\delta\v{r}(t)$ Note that $\v{r}$ is not necessarily generalized coordinate. $\v{r}(t)$ and $\v{r}^*(t)$ are on a surface to which the particle is constrained; this implies that the variation is also constrained. Now, we study the action:
\[S[\v{r}^* + \alpha\delta\v{r}] = \int_{t_1}^{t_2} \LL[\v{r}^* + \alpha\delta\v{r}, \dot{\v{r}}^* + \alpha\delta\dot{\v{r}}, t]dt\]
Now, we do a Taylor expansion around the true path:
\begin{align*}
    S[\v{r}^* + \alpha\delta\v{r}] &= \int_{t_1}^{t_2} dt\left(\LL(\v{r}^*, \dot{\v{r}}^*, t) + \alpha\delta\v{r}\dpd{\LL}{\v{r}} + \alpha\delta\dot{\v{r}}\dpd{\LL}{\dot{\v{r}}} + \delta(\alpha^2) \right) \quad \text{First term is unperturbed action, so we can write as:}
    \\ &= S[\v{r}^*, \dot{\v{r}}^*, t] + \alpha\int_{t_1}^{t_2}dt\dpd{\LL}{\v{r}}\delta\v{r} + \alpha\int_{t_1}^{t_2}dt\dpd{\LL}{\dot{\v{r}}}\dod{}{}t + \delta(\alpha^2)\quad\text{Integrating by parts, we have:}
    \\&= S[\v{r}^*, \dot{\v{r}}^*, t] + \alpha\int_{t_1}^{t_2}dt\dpd{\LL}{\v{r}}\delta\v{r} + \left. \dpd{\LL}{\dot{\v{r}}}\delta\v{r}\right|_{t_1}^{t_2} - \int_{t_1}^{t_2}dt\left(\dod{}{t}\dpd{\LL}{\dot{\v{r}}}\right)\delta\v{r}\quad\text{Boundary term is zero, so:}
    \\&= S[\v{r}^*, \dot{\v{r}}^*, t] + \alpha\int_{t_1}^{t_2}dt\left(\dpd{\LL}{\v{r}} - \dod{}{t}\dpd{\LL}{\dot{\v{r}}}\right)\delta\v{r} + \delta(\alpha^2)
\end{align*}
This is a first order variation (as we can tell from the fact that we expanded around $\alpha$ to first order). This is given by:
\[\delta S = \lim_{\alpha \rightarrow 0}\frac{1}{\alpha}\left(S[\v{r}^* + \alpha\delta\v{r}] -S[\v{r}^*]\right) = \int_{t_1}^{t_2}dt\left(\dpd{\LL}{\v{r}} - \dod{}{t}\dpd{\LL}{\dot{\v{r}}}\right)\delta\v{r}\]
We have that the first term $\dpd{\LL}{\v{r}} = -\grad{U}$ are conservative forces. Now considering the kinetic energy as $T = \frac{m}{2}\dot{\v{r}}^2$, then the second term is $\dod{}{t}\dpd{\LL}{\dot{\v{r}}} = m\ddot{\v{r}} = \v{F}_{tot} = \v{F}_{constraint} + \v{F}_{conservative}$. The contribution to the action integral for the constraint forces are given by $-\int_{t_1}^{t_2}\delta\v{r}\cdot\v{F}_{constaint} = 0$ as the constraint force is perpendicular to any direction for which we can actually very the path! Hence, the first order variation is given by:
\[\delta S = -\int_{t_1}^{t_2}\delta\v{r}\cdot\left(m\ddot{\v{r}} + \grad{U}\right)dt\]
If the particle follows Newton's laws of motion, then $m\ddot{\v{r}} + \grad{U} = 0$ and hence the first order variation is $\delta S = 0$! The beauty of this formulation is the constraint forces do not contribute. Here, $\delta\v{r}$ must be consistent with the constraint. So for the generalized coordinate, any variation works, and $\delta S[q] = 0$.
\subsubsection{Note on the Sign of Gravitational Potential Energy}
The sign in gravitational potential energy $U=mgy$ depends on whether the $y$-axis goes up or down. If you decide that the $y$-axis goes down then use $U=-mgy$, otherwise use $U=mgy$. To understand why, suppose we calculate the work done by gravity when $y$ goes up. This gives $W = \int_{y_1}^{y_2} -mg dy = +mgy_1 - mgy_2$, so we get the positive result. On the other hand, when $y$ goes down our work becomes $W=-mgy_1 + mgy_2$, so we get the negative result. \\
As an example, let's calculate $U$ for a pendulum blob. If $y$ goes up, the $y$ position of the blob is given by $y=-l\cos\theta$ and the gravitational potential is $U=mgy=-mgl\cos\theta$. If $y$ goes down, the $y$ position of the blob is given by $y=l\cos\theta$ and the gravitational potential is $U=-mgy=-mgl\cos\theta$. The result for $U$ is the same!